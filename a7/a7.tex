\documentclass[12pt]{article}

\usepackage{mathtools}
\DeclarePairedDelimiter{\ceil}{\lceil}{\rceil}

\usepackage{amsmath}
\usepackage[margin=1in]{geometry}
\usepackage{enumerate}
\usepackage{amssymb}
\usepackage{amsfonts}
\usepackage{pdfpages}
\usepackage[normalem]{ulem}
\usepackage[pdftitle={ECE 358 Assignment 7},%
pdfsubject={University of Waterloo, ECE 358},%
pdfauthor={Jason Shao, Lihao Luo, Minghao Lu}]{hyperref}

\title{ECE 358 Assignment 7}
\author{Jason Shao, Lihao Luo, Minghao Lu}
\date{July 25, 2016}

\begin{document}
\maketitle
\renewcommand{\thesubsection}{Problem \arabic{subsection}}


\def\question#1{\item[\bf #1.]}
\def\part#1{\item[\bf #1)]}
\newcommand{\pc}[1]{\mbox{\textbf{#1}}} % pseudocode

\begin{enumerate}
	\item
		Assuming loss rate is not 100 percent, then any data handed to the sender
		will eventually be received by the receiver intact(not corrupt). 
		(This sometimes considered a liveness property, but I asked the prof in class if we 
		can use it for this question and he said yes).
	\item
		In general, assuming sequence number is non repeating(infinite number of sequence
		numbers, prof confirmed valid assumption on piazza), YES. Receiver check sequence
		number and discard if not next expecting seq number, thus ensuring

		1) the packages are accepted in order \\
		2) if sender receives an ack, then all packages with seq number smaller or equal to seq 
		number in ack have been accepted by receiver. \\

		There is also special case we talked about in class where the receiver state machine
		breaks. The receiver state machine, when receiving a corrupt or out of order package,
		send back the last ack package it created. But the initialization of the  receiver
		state machine in the slides creates an ack for a package it hasn't received yet. 
		So if the very first package the receiver receives is an out of order package, it will
		send back an acknowledge to a package it hasn't received.

	\item
		Yes, similiar senario may occur if sender window size is the same as the number of sequence numbers.
		Assume sequence numbers are 0, 1 and window size is 2. Receiver can't distinguish between
		
		1) Sender sends 0,1. Receiver receives 0, send ack for 0. Receiver receives 1, 
		send ack for 1. Both acknowledgements are lost.  Sender times out, send 0,1 again. 
		Receiver receives retransmitted 0.\\
		2) Sender sends 0,1. Receiver receives 0, send ack for 0. Receiver receives 1,
		send ack for 1. Sender gets both ack. Sender sends new 0,1. Receiver receives new 0.

	\item
		a) Can't tell if an acknowledgement is from receiver receiving the original transmission or 
		an retransmission, thus can't estimate RRT for any packages that have been retransmitted.

		b) 3 Duplicate acks for the some sequence number means 3 packages with sequence number
		higher than that sequence number have been received, but the package with that sequence number 
		still haven't been received.  This implies the package with that sequence number was probably 
		lost in transit(rather than something like network is being slow), and therefore should be retransmitted.

	\item
		handling case 1) after connection terminates, server gets request. server sends ack to client, 
		since client is terminated, it doesn't send ack in response to server ack, server persumably times 
		out waiting for ack therefore doesn't open a "half open connection".

		handling case 2) after connection terminates, server gets request. server sends ack to client,
		since client is termianted, it doesn't send ack in response to server ack, no connection is
		established and data(x+1) is rejected

	\item
		If L is the loss rate, then the expectation is that 1/L packages are delivered before the first
		package is lost. Thus each cyle(increase cwnd from W/2 to W) delivers 1/L packages. (1)\\
		
		Since each successful roundtrip increases cwnd by one MSS. Let the unit of W be MSS, then
		$\frac{W}{2}$ roundtrips are required before cwnd increase from W/2 to W, completing one cyle. 
		Thus each period requires $\frac{W}{2}$ roundtrips. In slide 105, observe the number of packages 
		delivered in one cycle is the area under curver for one cycle. Thus number of packages delivered is\\
		$\frac{\frac{W}{2} + W}{2}_{MSS/roundtrip}* \frac{W}{2}_{roundtrips}=\frac{3}{8}W_{MSS}^2 $. (2)\\

		Setting (1) equal to (2) yields $W=\frac{\sqrt{\frac{8}{3}}}{\sqrt{L}}$MSS. Plugging this in: 
		$T=\frac{0.75W}{RTT}=\frac{0.75*\sqrt{\frac{3}{8}}MSS}{RTT*\sqrt{L}} \approx \frac{1.22MSS}{RTT*\sqrt{L}}$
\end{enumerate}
\end{document}
