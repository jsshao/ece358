\documentclass[12pt]{article}

\usepackage{mathtools}
\DeclarePairedDelimiter{\ceil}{\lceil}{\rceil}

\usepackage{amsmath}
\usepackage[margin=1in]{geometry}
\usepackage{enumerate}
\usepackage{amssymb}
\usepackage{amsfonts}
\usepackage[normalem]{ulem}
\usepackage[pdftitle={ECE 358 Assignment 4},%
pdfsubject={University of Waterloo, ECE 358},%
pdfauthor={Jason Shao, Lihao Luo, Minghao Lu}]{hyperref}

\title{ECE 358 Assignment 4}
\author{Jason Shao, Lihao Luo, Minghao Lu}
\date{June 23, 2016}

\begin{document}
\maketitle
\renewcommand{\thesubsection}{Problem \arabic{subsection}}


\def\question#1{\item[\bf #1.]}
\def\part#1{\item[\bf #1)]}
\newcommand{\pc}[1]{\mbox{\textbf{#1}}} % pseudocode

\begin{enumerate}
	\item %Q1
	\item %Q2 Jason
	During the first hop, the MTU is 1000 bytes, but the initial packet is 20 + 1800 = 1820 bytes. Therefore, after fragmentation, $f_1$ will have 20 bytes of header and 976 bytes of payload since the offset has to be a multiple of 8 while maximizing the total packet size to less than 1000 bytes.  Similarly, $f_2$ will have a header of 20 bytes and payload of the remaining 1800 - 976 = 824 bytes (offset of 122). \\ \\ Afterwards, $f_1$ undergoes fragmentation again with MTU of 500 bytes. The first part, $f_1.1$ will have 20 bytes for the header and 480 bytes for payload. The second part, $f_1.2$ will have 20 bytes for the header and 480 bytes for payload (offset of 60). The third part, $f_1.3$ will have 20 bytes for the header and 976 - 480 - 480 = 16 bytes for the payload (offset of 120). \\ \\ In conclusion, the final fragments received at the destination in order of offset is:
	\begin{itemize}
		\item First fragment: ID = abcd, More fragments = 1, Fragment offset = 0, Total length = 500 bytes (480 bytes of payload)
		\item Second fragment: ID = abcd, More fragments = 1, Fragment offset = 60, Total length = 500 bytes (480 bytes of payload)
		\item Third fragment: ID = abcd, More fragments = 1, Fragment offset = 120, Total length = 36 bytes (16 bytes of payload)
		\item Fourth fragment: ID = abcd, More fragments = 0, Fragment offset = 122, Total length = 844 bytes (824 bytes of payload)
	\end{itemize}
	\item %Q3 Jason
	\item %Q4 Frank
	\item %Q5 Frank
	\item %Q6 Frank
	\item %Q7 Jason

\end{enumerate}
\end{document}
