\documentclass[12pt]{article}

\usepackage{mathtools}
\DeclarePairedDelimiter{\ceil}{\lceil}{\rceil}

\usepackage{amsmath}
\usepackage[margin=1in]{geometry}
\usepackage{enumerate}
\usepackage{amssymb}
\usepackage{amsfonts}
\usepackage[normalem]{ulem}
\usepackage[pdftitle={ECE 358 Assignment 4},%
pdfsubject={University of Waterloo, ECE 358},%
pdfauthor={Jason Shao, Lihao Luo, Minghao Lu}]{hyperref}

\title{ECE 358 Assignment 4}
\author{Jason Shao, Lihao Luo, Minghao Lu}
\date{June 23, 2016}

\begin{document}
\maketitle
\renewcommand{\thesubsection}{Problem \arabic{subsection}}


\def\question#1{\item[\bf #1.]}
\def\part#1{\item[\bf #1)]}
\newcommand{\pc}[1]{\mbox{\textbf{#1}}} % pseudocode

\begin{enumerate}
	\item %Q1
	\item %Q2 Jason
	During the first hop, the MTU is 1000 bytes, but the initial packet is 20 + 1800 = 1820 bytes. Therefore, after fragmentation, $f_1$ will have 20 bytes of header and 976 bytes of payload since the offset has to be a multiple of 8 while maximizing the total packet size to less than 1000 bytes.  Similarly, $f_2$ will have a header of 20 bytes and payload of the remaining 1800 - 976 = 824 bytes (offset of 122). \\ \\ Afterwards, $f_1$ undergoes fragmentation again with MTU of 500 bytes. The first part, $f_1.1$ will have 20 bytes for the header and 480 bytes for payload. The second part, $f_1.2$ will have 20 bytes for the header and 480 bytes for payload (offset of 60). The third part, $f_1.3$ will have 20 bytes for the header and 976 - 480 - 480 = 16 bytes for the payload (offset of 120). \\ \\ In conclusion, the final fragments received at the destination in order of offset is:
	\begin{itemize}
		\item First fragment: ID = abcd, More fragments = 1, Fragment offset = 0, Total length = 500 bytes (480 bytes of payload)
		\item Second fragment: ID = abcd, More fragments = 1, Fragment offset = 60, Total length = 500 bytes (480 bytes of payload)
		\item Third fragment: ID = abcd, More fragments = 1, Fragment offset = 120, Total length = 36 bytes (16 bytes of payload)
		\item Fourth fragment: ID = abcd, More fragments = 0, Fragment offset = 122, Total length = 844 bytes (824 bytes of payload)
	\end{itemize}
	\item %Q3 Jason
	\item %Q4 Frank
        Like the assignment suggested, we adopt two premises. 
        (i) at the point in time the slide considers, for every $a \in N'$, $D(a) = d(a)$. 
        (ii) The path $u \rightsquigarrow y$ in the picture, $u \rightsquigarrow x \rightarrow y$,
        is a cheapest path from u to y. Let's call this path $p_1$.\\


        First we prove $D(y) \leq cost(p_1)$. By definition $cost(p_1) = d(x) + c(x,y)$. Since $x \in N'$, 
        by premise (i), $cost(p_1) = D(x) + c(x,y)$. When x was added to $N'$, since y is adjacent to x,
        the algorithm performs $D(y) = min\{D(y), D(x)+c(x,y)\}$, and since the only operations performed on $D(y)$ is to assign
        it a min of its old value and another value, $D(y)$ never increases. Thus $D(y) \leq D(x) + c(x,y)$,
        which combined with $cost(p_1) = D(x) + c(x,y)$, implies $D(y) \leq cost(p_1)$. \\

        Suppose $D(y) \neq d(y)$. Since $d(y)$ is the cheapest cost from u to y, and $D(y)$ is the 
        cost of a path from u to y, $D(y) \geq d(y)$. Since $D(y) \neq d(y)$, $D(y) > d(y)$. 
        Since $d(y) < D(y)$ and $D(y) \leq cost(p_1)$, there must be a path from u to y that is cheaper 
        than $p_1$. But premise (ii) says $p_1$ is a cheapest path from u to y, contradiction. Therefor $D(y) = d(y)$.
        

	\item %Q5 Frank
        
        First we observe $\min_{v \in neigh(x)}\{c(x,v) + d_v(y)\}$
        is at least an upper bound on $d_x(y)$. Let $v_1$ be a neighbour of x that achieves the minimum in
        $\min_{v \in neigh(x)}\{c(x,v) + d_v(y)\}$, i.e $c(x,v_1)+d_{v_1}(y) = \min_{v \in neigh(x)}\{c(x,v) + d_v(y)\}$.  
        Let $v_1 \rightsquigarrow y$ be a minimum path from $v_1$
        to y, i.e. $cost(v_1 \rightsquigarrow y) = d_{v_1}(y)$. Observe $x \rightarrow v_1 \rightsquigarrow y$ is a path from x to y. Moreover
        $cost(x \rightarrow v_1 \rightsquigarrow y) = c(x, v_1) + cost(v_1 \rightsquigarrow y) = c(x, v_1) + d_{v_1}(y) =
        \min_{v \in neigh(x)}\{c(x,v) + d_v(y)\})$.
        Since $d_x(y)$ is the cost of the cheapest path from x to y, 
        $d_x(y) \leq cost(x,v_1 \rightsquigarrow y) = \min_{v \in neigh(x)}\{c(x,v) + d_v(y)\}$. \\

        Suppose $d_x(y) = \min_{v \in neigh(x)}\{c(x,v) + d_v(y)\}$ is not true, since we proved
        $d_x(y) \leq \min_{v \in neigh(x)}\{c(x,v) + d_v(y)\}$, it must be the case 
        $d_x(y) < \min_{v \in neigh(x)}\{c(x,v) + d_v(y)\}$.
        Let $p=x,v_1,v_2,...v_k,y$ be a cheapest path from x to y, then  
        $cost(p) = c(x,v_1) + cost(v_1, v_2,...,v_k,y) < \min_{v \in neigh(x)}\{c(x,v) + d_v(y)\}$. 
        In particular, $c(x, v_1) + cost(v_1,...,v_k,y) < c(x, v_1) + d_{v_1}(y)$,
        which in turn implies $cost(v_1,...,v_k,y) < d_{v_1}(y)$.
        Since $v_1,...,v_k,y$ is a path from $v_1$ to y and $d_{v_1}(y)$ is the cost of the cheapest path from 
        $v_1$ to y, this is a contradiction. Therefor our assumption, $d_x(y) = \min_{v \in neigh(x)}\{c(x,v) + d_v(y)\}$ is not true,
        must be false.

	\item %Q6 Frank

        The number of nodes in G.

        Let $v_1,...,v_k$ be the cheapest path between $v_1,...,v_k$.  First we observe
        $\forall i \leq k, v_1, v_2, ..., v_i$ is the cheapest path between $v_1$ and $v_i$.
        Suppose its not, then there must be path $v_1,w_1,w_2,...,v_i$ that's cheaper, but then
        $v_1,w_1,w_2,...,v_i,v_{i+1},v_{i+2},...,v_k$ would be cheaper than $v_1,v_2,...,v_k$.
        Since we defined $v_1,v_2,...,v_k$ as the cheapest path between $v_1$ and $v_k$, this is a
        contradiction. Thus $\forall i \leq k: v_1, v_2, ..., v_i$ must be the cheapest path between $v_1$ and $v_i$. \\

        (For the following section, by "know" the cheapest path I mean know the next hop and total cost) \\

        In an iteration, if $v_i$ doesn't know its cheapest path from $v_1$ yet, since $v_i$ updates its guess of the cheapest
        from $v_1$ path based only on its neighbours guess of cheapest path from $v_1$,
        $v_i$ would and would only learn its true cheapest path from $v_1$ if in the last 
        iteration $v_{i-1}$ knows its true cheapest path from $v_1$. Since by the end of the first iteration, only
        $v_1$ will know its distance for $v_1$, it takes a total of i iterations for $v_i$ to learn
        its cheapest path from $v_1$. Thus it takes k iterations for $v_k$ to learn its cheapest path
        from $v_1$.

        It follows that the number of iterations for everyone to know their cheapest path from everyone else 
        will be the number of nodes in the longest path between any two nodes where that path is also the 
        cheapest path between the two nodes. Such a path in a weighted, connected, undirected graph G
        could contain all the nodes in G.



	\item %Q7 Jason

\end{enumerate}
\end{document}
