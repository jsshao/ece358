\documentclass[12pt]{article}

\usepackage{mathtools}
\DeclarePairedDelimiter{\ceil}{\lceil}{\rceil}

\usepackage{amsmath}
\usepackage[margin=1in]{geometry}
\usepackage{enumerate}
\usepackage{amssymb}
\usepackage{amsfonts}
\usepackage[normalem]{ulem}
\usepackage[pdftitle={ECE 358 Assignment 3},%
pdfsubject={University of Waterloo, ECE 358},%
pdfauthor={Jason Shao, Lihao Luo, Minghao Lu}]{hyperref}

\title{ECE 358 Assignment }
\author{Jason Shao, Lihao Luo, Minghao Lu}
\date{June 5, 2016}

\begin{document}
\maketitle
\renewcommand{\thesubsection}{Problem \arabic{subsection}}


\def\question#1{\item[\bf #1.]}
\def\part#1{\item[\bf #1)]}
\newcommand{\pc}[1]{\mbox{\textbf{#1}}} % pseudocode

\begin{enumerate}
    \item The tight lower bound for the largest-sized routing table across all the $n$ routers is 2. This is because a router in this IP network forwards packets between exactly 2 subnetworks. That means it has to be able to lookup at least two different subnet prefixes to know how to forward the datagram to each of the two subnetworks. \\ \\
    \item 
    \begin{verbatim}
    def decision(P, k):
        P' = P
        for i <- 0 to P'.size() - 1:
            for j <- i + 1 to P'.size() - 1:
                if mergeable(P'[i], P'[j]):
                    P'.add(merge(P'[i], P'[j]))
                    P'.remove(P'[i])
                    P'.remove(P'[j])
                    i <- 0
                    j <- 0
        return P'.size() <= k    					
    \end{verbatim}
  
	\item Mike's go here
	\item Yes, I concur with Alice. \\ \\Let $T$ be the initial minimum spanning tree of $G$. Let $u$ be the node that leaves $G$, and let $v$ be the node such that $uv$ is the only edge incident to $u$ in $T$. Let $T'$ be $T$ without $u$ and $uv$. I want to show that $T'$ is the minimum spanning tree of the new graph. \\ \\ \textbf{Proof By Contradiction} \\ Assume for the sake of contradiction that $T'$ is not the most minimum spanning tree of the new graph. That means there must exist some tree $R$ that spans the new graph, and the total weight of its edges is less than that of $T'$. \\ \\ Now consider $R$ with $uv$ added to the tree. We can show that $R$ with $uv$ is a spanning tree of $G$, because $R$ is a tree that spans every node in $G$ except $u$. However, we know $v$ is in $R$ so $uv$ can be added to $R$ to span all nodes in $G$ while maintaining the tree structure. \\ \\ Moreover, observe that $R$ with $uv$ has total weight less than that of $T$ because:
	\begin{align*}
	weight(R) &< weight(T')\\
	weight(R) + weight(uv) &< weight(T') + weight(uv)\\
	weight(R) + weight(uv) &< weight(T)
	\end{align*}
	Now we have shown that $R$ with $uv$ is a spanning tree of $G$ and its weight is less than that of the $T$, the minimum spanning tree. This is clearly a contradiction, so we can declare that there does not exist another spanning tree that has total weight less than $T'$. Therefore, we do not need to recompute the MST as when a node leaves the network and it only has one edge incident on it, the new MST is merely the original tree with the node and its incident edge removed.
	\item
        According to the book, for a request there are two "receiver" MAC address in the frame:
        DST address in the header that is 48 bits of 1 or Target address in the message
        that is just 0. I'm going to assume you want the latter. \\

        part a) \\
        (i) request (ii) 1.2.3.4 (iii) 1.2.3.4's MAC address (iv) 1.2.3.10 (v) 0 \\
        (i) response (ii) 1.2.3.10 (iii) 1.2.3.10's MAC address (iv) 1.2.3.4 (v) 1.2.3.4's MAC address \\

        part b) \\
        (i) request (ii) 1.2.3.4 (iii) 1.2.3.4's MAC address (iv) 1.2.1.1 (v) 0 \\
        (i) response (ii) 1.2.1.1 (iii) 1.2.1.1's MAC address (iv) 1.2.3.4 (v) 1.2.3.4's MAC adress \\
        (i) request (ii) 10.11.12.1 (iii) 10.11.12.1's MAC address (iv) 10.11.12.25 (v) 0 \\
        (i) response (ii) 10.11.12.25 (iii) 10.11.12.25's MAC address (iv) 10.11.12.1 (v) 10.11.12.1's MAC address \\
        (i) request (ii) 15.16.17.25 (iii) 15.16.17.25's MAC address (iv) 15.16.17.18 (v) 0 \\
        (i) response (ii) 15.16.17.18 (iii) 15.16.17.18's MAC address (iv) 15.16.17.25 (v) 15.16.17.25's MAC address \\

\end{enumerate}
\end{document}
