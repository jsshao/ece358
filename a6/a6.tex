\documentclass[12pt]{article}

\usepackage{mathtools}
\DeclarePairedDelimiter{\ceil}{\lceil}{\rceil}

\usepackage{amsmath}
\usepackage[margin=1in]{geometry}
\usepackage{enumerate}
\usepackage{amssymb}
\usepackage{amsfonts}
\usepackage[normalem]{ulem}
\usepackage[pdftitle={ECE 358 Assignment 3},%
pdfsubject={University of Waterloo, ECE 358},%
pdfauthor={Jason Shao, Lihao Luo, Minghao Lu}]{hyperref}

\title{ECE 358 Assignment }
\author{Jason Shao, Lihao Luo, Minghao Lu}
\date{July 15, 2016}

\begin{document}
\maketitle
\renewcommand{\thesubsection}{Problem \arabic{subsection}}


\def\question#1{\item[\bf #1.]}
\def\part#1{\item[\bf #1)]}
\newcommand{\pc}[1]{\mbox{\textbf{#1}}} % pseudocode

\begin{enumerate}
\item The route taken by the packet from $A$ to $B$ is $[A, R_1, R_4, R_5, R_6, B]$. 

First, the AS that contains $A$ ($AS_A$) learns from the inter-AS protocol that $B$ is reachable through both $AS_B$ and $AS_{R4}$. Since the intra-AS protocol used is hot potato routing, then the gateway router chosen by $A$ is the one with the least hops. This happens to be $R_1$. Therefore, the packet is forwarded to $R_1$ to $AS_{R4}$. The rest of the routing follows the same logic and the final path becomes $[A, R_1, R_4, R_5, R_6, B]$.

\item X would advertise 1.2.0.0/16 and 5.6.7.0/24 to both B and C.

The two connecting subnets are not part of X's responsibility. X as
a client shouldn't have to worry about them. The subnet X bought and uses X does have to advertise.

\end{enumerate}
\end{document}
