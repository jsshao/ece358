\documentclass[12pt]{article}

\usepackage{mathtools}
\DeclarePairedDelimiter{\ceil}{\lceil}{\rceil}

\usepackage{amsmath}
\usepackage[margin=1in]{geometry}
\usepackage{enumerate}
\usepackage{amssymb}
\usepackage{amsfonts}
\usepackage[normalem]{ulem}
\usepackage[pdftitle={ECE 358 Assignment 2},%
pdfsubject={University of Waterloo, ECE 358},%
pdfauthor={Jason Shao, Lihao Luo, Minghao Lu}]{hyperref}

\title{ECE 358 Assignment 1}
\author{Jason Shao, Lihao Luo, Minghao Lu}
\date{May 21, 2016}

\begin{document}
\maketitle
\renewcommand{\thesubsection}{Problem \arabic{subsection}}


\def\question#1{\item[\bf #1.]}
\def\part#1{\item[\bf #1)]}
\newcommand{\pc}[1]{\mbox{\textbf{#1}}} % pseudocode

\begin{enumerate}
    \item 
    Suppose we are given some integer $m > 1$, which parameterizes the $m$-bit ID's of our Chord DHT. Then, consider an instance of the DHT that is the following:
    \begin{itemize}
        \item A peer at 0
        \item A peer at $2^{m-1}$
        \item A peer at $2^{m} - 1$
    \end{itemize}
    
    \begin{figure}[!ht]
    \centering
    \includegraphics[width=0.5\textwidth]{Chord}
    \caption{Configuration of Chord}
	\end{figure}
    As we can see from Figure 1, the finger table for peer 0 contains all $2^{m-1}$. This is because $FT_p[i] = succ(p + 2^{i-1}) , i \in [1,m]$ so $$FT_0[i] = succ(0 + 2^{i-1}), i \in [1, m]$$
    This ranges from $succ(0 + 2^0)$ to $succ(0 + 2^{m-1})$ which both have values of peer $2^{m-1}$ since there are no peers between peer 0 and peer $2^{m-1}$. Hence, the finger table values of peer 0 are all $2^{m-1}$. \\ \\
    Now, consider $lookup(2^m - 1)$ at peer 0. We know that the next hop will always be peer $2^{m-1}$. In this situation, $p = 0$, $q = 2^{m-1}$, $k = 2^m - 1$. We can see that there are no peers between $p$ and $q$. However, there is one peer between $p$ and $k$. Therefore, we can draw the following contradiction from the original claim:
    \begin{align*}
    q - p &\ge (k-p)/2 \\
    0 &\ge \frac{1}{2}
    \end{align*}
    By the principle of contradiction, it is shown for all $m > 1$, there exists some configuration that disproves $ q - p \ge (k-p)/2 $.
    
    \item 
    \begin{enumerate}
        \item  \underline{$N_2$} \\ \\
        \begin{tabular}{ |c|c|c| } 
         \hline
         & Binary Form & Dot Decimal Notation \\ 
         \hline
         IP Address & 00000001.00000010.00000011.00000100 & 1.2.3.4 \\ 
         Subnet Mask & 11111111.11111111.11111111.11110000 & 255.255.255.240 \\
         After Mask & 00000001.00000010.00000011.00000000 & 1.2.3.0 \\
         \hline
        \end{tabular}
        
        We see that address 1.2.3.4 does not lie in $N_1$ as the post-mask value is not 1.2.3.160.\\ \\   
        \begin{tabular}{ |c|c|c| } 
         \hline
         & Binary Form & Dot Decimal Notation \\ 
         \hline
         IP Address & 00000001.00000010.00000011.00000100 & 1.2.3.4 \\ 
         Subnet Mask & 11111111.11111111.11111111.00000000 & 255.255.255.0 \\
         After Mask & 00000001.00000010.00000011.00000000 & 1.2.3.0 \\
         \hline
        \end{tabular}
        
        We see that address 1.2.3.4 does lie in $N_2$ as the post-mask value is 1.2.3.0. \\
        
        \item \underline{$N_2$}\\ \\
        \begin{tabular}{ |c|c|c| } 
         \hline
         & Binary Form & Dot Decimal Notation \\ 
         \hline
         IP Address & 00000001.00000010.00000011.11000011 & 1.2.3.195 \\ 
         Subnet Mask & 11111111.11111111.11111111.11110000 & 255.255.255.240 \\
         After Mask & 00000001.00000010.00000011.11000000 & 1.2.3.192 \\
         \hline
        \end{tabular}
        
        We see that address 1.2.3.195 does not lie in $N_1$ as the post-mask value is not 1.2.3.160.\\ \\
        \begin{tabular}{ |c|c|c| } 
         \hline
         & Binary Form & Dot Decimal Notation \\ 
         \hline
         IP Address & 00000001.00000010.00000011.11000011 & 1.2.3.195 \\ 
         Subnet Mask & 11111111.11111111.11111111.00000000 & 255.255.255.0 \\
         After Mask & 00000001.00000010.00000011.00000000 & 1.2.3.0 \\
         \hline
        \end{tabular}
        
        We see that address 1.2.3.195 does lie in $N_2$ as the post-mask value is 1.2.3.0. \\
        
        \item \underline{Both}\\ \\
        \begin{tabular}{ |c|c|c| } 
         \hline
         & Binary Form & Dot Decimal Notation \\ 
         \hline
         IP Address & 00000001.00000010.00000011.10101011 & 1.2.3.171 \\ 
         Subnet Mask & 11111111.11111111.11111111.11110000 & 255.255.255.240 \\
         After Mask & 00000001.00000010.00000011.10100000 & 1.2.3.160 \\
         \hline
        \end{tabular}
        
        We see that address 1.2.3.171 does does lie in $N_1$ as the post-mask value is 1.2.3.160. \\ \\
        \begin{tabular}{ |c|c|c| } 
         \hline
         & Binary Form & Dot Decimal Notation \\ 
         \hline
         IP Address & 00000001.00000010.00000011.10101011 & 1.2.3.171 \\ 
         Subnet Mask & 11111111.11111111.11111111.00000000 & 255.255.255.0 \\
         After Mask & 00000001.00000010.00000011.00000000 & 1.2.3.0 \\
         \hline
        \end{tabular}
        
        We see that address 1.2.3.171 does lie in $N_2$ as the post-mask value is 1.2.3.0.
    \end{enumerate}
\end{enumerate}
\end{document}
